\item 
	\begin{enumerate}[a)]
		\item Considere la siguiente ecuación de movimiento:
		\begin{equation*}
			a \ddot{x} + b \dot{x} + cx = d
		\end{equation*}
		donde $a$, $b$, $c$ y $d$ son constantes. En función de estas constantes determine:
		\begin{enumerate}
			\item la posición de equilibrio $x_{\text{eq}}$.
					
			\item la constante de amortiguamiento $\gamma$.
			
			\item la frecuencia natural $\omega_0$.
			
			\item ¿Qué condición tienen que satisfacer las constantes para que $x(t)$ sea oscilante?. Una vez encontrada la condición, determine la frecuencia de oscilación del sistema.
		\end{enumerate}
		
		\bigskip
		
		\item Considere la siguiente ecuación de movimiento:
		\begin{equation*}
			A \ddot{x} + B \dot{x} + Cx = Dt + E
		\end{equation*}
		donde $A$, $B$, $C$, $D$ y $E$ son constantes.
		\begin{enumerate}
			\item Determine la solución homogénea $x_h(t)$ para el amortiguamiento débil en función de las constantes $A$, $B$, $C$, $D$ y $E$.
			
			\item Si la solución particular es de la forma $x_p(t) = c_1 t + c_2$. Determine las constantes $c_1$ y $c_2$ en función de las constantes $A$, $B$, $C$, $D$ y $E$.
			
			\item Grafique de forma general la solución homogénea $x_h(t)$, la solución particular $x_p(t)$ y la solución $x(t) = x_h(t) + x_p(t)$. Realice las tres curvas en el siguiente gráfico indicando el nombre de cada curva.
		\end{enumerate}
	\end{enumerate}

\underline{Soluci\'on:}

\begin{enumerate}[a)]
	\item
	\begin{enumerate}
		\item La ecuación de movimiento es:
		\begin{equation*}
			\ddot{x} + \dfrac{b}{a} \dot{x} + \dfrac{c}{a} x = \dfrac{d}{a}
		\end{equation*}
		
		La posición de equilibrio es:
		\begin{align*}
			\dfrac{c}{a} x_{\text{eq}} &= \dfrac{d}{a} \\
			x_{\text{eq}} &= \frac{d}{c}
		\end{align*}
	
		\item 
		\begin{align*}
			2 \gamma = \frac{b}{a} \\
			\gamma = \frac{b}{2a}
		\end{align*}
	
		\item
		\begin{align*}
			\omega_0^2 = \frac{c}{a} \\
			\omega_0 = \sqrt{\frac{c}{a}}
		\end{align*}
	
		\item
		\begin{align*}
			\omega &= \sqrt{ \omega_0^2 - \gamma^2 } \\
			\omega &= \sqrt{ \dfrac{c}{a} - \dfrac{b^2}{4a^2} } \\
			\omega &= \dfrac{1}{2a} \sqrt{ 4ac - b^2}
		\end{align*}
		La condición para que exista oscilación es $4ac - b^2 > 0$.		
	\end{enumerate}

	\item
	\begin{enumerate}
		\item La ecuación de movimiento homogénea es:
		\begin{equation*}
			\ddot{x} + \dfrac{B}{A} \dot{x} + \dfrac{C}{A} x = 0
		\end{equation*}
		
		La solución homogénea $x_h(t)$ es de la forma:
		\begin{equation*}
			x_h(t) = A_1 e^{-\gamma t} \cos \left( \omega_1 t + \phi_1 \right)
		\end{equation*}
		
		De forma análoga al ítem anterior, obtenemos las frecuencias $\omega_0$, $\gamma$ y $\omega_1$ como:
		\begin{align*}
			\omega_0 &= \sqrt{\frac{C}{A}} \\
			\gamma &= \frac{B}{2A} \\
			\omega_1 &= \sqrt{ \frac{C}{A} - \frac{B^2}{4A^2}}
		\end{align*}
		
		Por lo tanto, la solución es:
		\begin{equation*}
			x_h(t) = A_1 e^{-\frac{B}{2A} t} \cos \left( \sqrt{ \frac{C}{A} - \frac{B^2}{4A^2}} \ t + \phi_1 \right)
		\end{equation*}
		donde $A_1$ y $\phi_1$ se obtienen con las condiciones iniciales del problema.
		
		\item 
		\begin{align*}
			x_p(t) &= c_1 t + c_2 \\
			\dot{x}_p(t) &= c_1 \\
			\ddot{x}_p(t) &= 0
		\end{align*}
		
		Reemplazando en la ecuación de movimiento obtenemos:
		\begin{align*}
			A \ddot{x} + B \dot{x} + Cx &= Dt + E \\
			B c_1 + C \left( c_1 t + c_2 \right) &= Dt + E
		\end{align*}
		
		Obtenemos dos ecuaciones:
		\begin{align*}
			C c_1 &= D \\
			B c_1 + C c_2 &= E
		\end{align*}
		
		Las constantes son:
		\begin{align*}
			c_1 &= \frac{D}{C} \\
			c_2 &= \frac{CE-BD}{C^2}
		\end{align*}
	
		\item
		\begin{align*}
			x_h(t) &= A_1 e^{-\frac{B}{2A} t} \cos \left( \sqrt{ \frac{C}{A} - \frac{B^2}{4A^2}} \ t + \phi_1 \right) \\
			x_p(t) &= \frac{D}{C}t + \frac{CE-BD}{C^2} \\
			x(t) &= A_1 e^{-\frac{B}{2A} t} \cos \left( \sqrt{ \frac{C}{A} - \frac{B^2}{4A^2}} \ t + \phi_1 \right) + \frac{D}{C}t + \frac{CE-BD}{C^2}
		\end{align*}
		
		\begin{center}
			\begin{tikzpicture}[scale = 1]
			\draw[->, >=latex] (0,0) -- (15,0);
			\draw[->, >=latex] (0,-6) -- (0,6);
			\node at (15.5,0) {\(t\)};
			\node at (-0.2,6.2) {\(x\)};
			
			\draw [domain=0:15, samples = 500] plot(\x, { 0.2*\x + 2} );
			\draw [domain=0:15, samples = 500] plot(\x, { -3*exp(-.2*\x) * cos(2*\x r ) } );
			\draw [domain=0:15, samples = 500] plot(\x, { 0.2*\x + 2 - 3*exp(-.2*\x) * cos(2*\x r ) } );
			
			\node at (6.2,-1.2) {\(x_h(t)\)};
			\node at (3.3,3) {\(x_p(t)\)};
			\node at (1.5,4.8) {\(x(t)\)};	
			\end{tikzpicture}
		\end{center}
	\end{enumerate}
\end{enumerate}
