\item
	Considere la siguiente función $f(x,t)$, donde $A$, $k$ y $w$ son constantes; la posición $x$ se mide en metros y el tiempo $t$ en segundos.
	\begin{equation*}
	f(x,t) = A \ \dfrac{e^{kx-1}}{e^{wt-1}} \left( \cos(kx) \cos(wt) + \sin(kx) \sin(wt) \right)
	\end{equation*}
	
\begin{enumerate}[a)]
	\item Demuestre que la función $f(x,t)$ corresponde a una onda.
	
	\item Si $k=2$ [rad/m] y $w=3$ [rad/s], determine la velocidad de propagación de la onda.

	\item Un pulso rectangular de amplitud $A=1$ [m] y largo $L=3$ [m] tiene una velocidad de propagación de $1$ [m/s] en un medio de densidad $\rho_1$. Si en $t=0$ este pulso de onda choca con otro medio de densidad $\rho_2$, $\rho_2 \gg \rho_1$, dibuje la onda total en los tiempos $t=0, 1, 2, 3$ [s].
\end{enumerate}

\textbf{\underline{Solución:}}

\begin{enumerate}[a)]
	
	\item La ecuación de la onda es: \[ \frac{\partial^2f}{\partial x^2} = \frac{1}{v^2} \ \frac{\partial^2f}{\partial t^2} \] donde \( v = \lambda f \), pero \( K = 2\pi / \lambda  \Rightarrow v = 2\pi f / K \). \\
	Además \( \omega = 2\pi f  \Rightarrow f = \omega / 2\pi \Rightarrow v = \omega / K \). Por lo tanto se tiene que:
	\[ \frac{1}{v^2} = \frac{K^2}{\omega^2} \]
	
	Por otro lado:
	\begin{align*}
		f(x,t) &= A \cdot K \cdot \frac{e^{Kx-1}}{e^{\omega t-1}} \cdot \cos(Kx-\omega t)\\
		\frac{\partial f}{\partial x} &= A \cdot K \cdot \frac{e^{Kx-1}}{e^{\omega t-1}} \cdot \cos(Kx-\omega t) - A \cdot K \cdot \frac{e^{Kx-1}}{e^{\omega t-1}} \cdot \sin(Kx - \omega t) \\
		\frac{\partial f}{\partial x} &= A \cdot K \cdot \frac{e^{Kx-1}}{e^{\omega t-1}} \cdot [\cos(Kx-\omega t) - \sin(Kx-\omega t)] \\
		\frac{\partial^2 f}{\partial x^2} &= A K^2 \cdot \frac{e^{Kx-1}}{e^{\omega t-1}}\left[ \cancel{\cos (Kx-\omega t)} - \sin (Kx-\omega t) - \sin (Kx-\omega t) - \cancel{\cos (Kx-\omega t)} \right]\\
		\frac{\partial^2 f}{\partial x^2} &= -2A K^2 \cdot \frac{e^{Kx-1}}{e^{\omega t-1}} \cdot \sin (Kx-\omega t)\\
		\frac{\partial f}{\partial t} &= -A \cdot \omega \cdot \frac{e^{Kx-1}}{e^{\omega t-1}} \left[ cos(Kx-\omega t) - \sin(Kx-\omega t) \right]\\
		\frac{\partial^2 f}{\partial t^2} &= A \omega^2 \cdot \frac{e^{Kx-1}}{e^{\omega t-1}}\left[ \cancel{\cos (Kx-\omega t)} - \sin (Kx-\omega t) - \sin (Kx-\omega t) - \cancel{\cos (Kx-\omega t)} \right]
	\end{align*}	
	
	Reemplazando en la ecuación de onda
	\begin{align*}
	-\cancel{2}\cancel{K^2}\cancel{A}\cancel{\frac{e^{Kx-1}}{e^{\omega t-1}}}\cdot\cancel{\sin(Kx- \omega t)} &= \frac{\cancel{K^2}}{\cancel{\omega^2}} \cdot -\cancel{2}\cancel{\omega^2}\cancel{A}\cancel{\frac{e^{Kx-1}}{e^{\omega t-1}}} \cdot\cancel{\sin(Kx- \omega t)} \\
	1 &= 1
	\end{align*}
	Por lo tanto se comprueba que \( f(x,t) \) es una onda.
	\item De lo mencionado anteriormente
	\begin{equation*}
		\tcbhighmath{v = \frac{\omega}{K} = \frac{3\left[ \text{rad/s} \right]}{2\left[ \text{rad/m} \right]} = 1.5[\text{m/s}]}
	\end{equation*}
\end{enumerate}

Como \(\rho_2 >>> \rho_1\), no existe transmisión del pulso al medio 2, por lo cual éste se refleja completamente con una amplitud \(-A\), siendo \(A\) la amplitud de la onda incidente

\begin{figure}[H]
	\centering
	\begin{tikzpicture}
		\begin{scope}
			\node[scale=1.3] at (-3,3) {\( t = 0 [\text{s}] \)};
			\draw[dashed] (-4,0) -- (2,0);
			\draw[dashed] (0,2.5) -- (0,-2.5);
			\draw[ultra thick] (-4,0) -- (-3,0) -- (-3,1) -- (0,1);
			\draw[|-|,thick] (0.5,1) -- (0.5,0) node[midway,right,scale=0.9] {\( 1 [\text{m} ] \)};
			\draw [|-|,thick] (-3,-0.5) -- (0,-0.5) node [midway,below] {\( L = 3[\text{m}] \)};
			\node [scale=1.4] at (-2,2.3) {\( \rho_1 \)};
			\node [scale=1.4] at (1.5,2.3) {\( \rho_2 \)};
			\draw [->,thick] (-2,1.5) -- (-0.5,1.5) node[above,midway] {\( v \)};
		\end{scope}
	
		\begin{scope}[shift={(7,0)}]
			\node[scale=1.3] at (-3,3) {\( t = 1 [\text{s}] \)};
			\draw[dashed] (-4,0) -- (2,0);
			\draw[dashed] (0,2.5) -- (0,-2.5);
			\draw[ultra thick] (-4,0) -- (-2,0) -- (-2,1) -- (-1,1) -- (-1,0) -- (0,0);
			\draw[|-|,thick] (0.5,1) -- (0.5,0) node[midway,right,scale=0.9] {\( 1 [\text{m} ] \)};
			\draw [|-|,thick] (-2,-0.5) -- (-1.02,-0.5) node [midway,below] {\( 1[\text{m}] \)};
			\draw [|-|,thick] (-1,-0.5) -- (0,-0.5) node [midway,below] {\( 1[\text{m}] \)};
			\node [scale=1.4] at (-2,2.3) {\( \rho_1 \)};
			\node [scale=1.4] at (1.5,2.3) {\( \rho_2 \)};
		\end{scope}
	
		\begin{scope}[shift={(0,-7)}]
			\node[scale=1.3] at (-3,3) {\( t = 2 [\text{s}] \)};
			\draw[dashed] (-4,0) -- (2,0);
			\draw[dashed] (0,2.5) -- (0,-2.5);
			\draw[ultra thick] (-4,0) -- (-2,0) -- (-2,-1) -- (-1,-1) -- (-1,0) -- (0,0);
			\draw[|-|,thick] (0.5,-1) -- (0.5,0) node[midway,right,scale=0.9] {\( 1 [\text{m} ] \)};
			\draw [|-|,thick] (-2,0.5) -- (-1.02,0.5) node [midway,above] {\( 1[\text{m}] \)};
			\draw [|-|,thick] (-1,0.5) -- (0,0.5) node [midway,above] {\( 1[\text{m}] \)};
			\node [scale=1.4] at (-2,2.3) {\( \rho_1 \)};
			\node [scale=1.4] at (1.5,2.3) {\( \rho_2 \)};
		\end{scope}
	
		\begin{scope}[shift={(7,-7)}]
			\node[scale=1.3] at (-3,3) {\( t = 3 [\text{s}] \)};
			\draw[dashed] (-4,0) -- (2,0);
			\draw[dashed] (0,2.5) -- (0,-2.5);
			\draw[ultra thick] (-4,0) -- (-3,0) -- (-3,-1) -- (0,-1);
			\draw[|-|,thick] (0.5,-1) -- (0.5,0) node[midway,right,scale=0.9] {\( 1 [\text{m} ] \)};
			\draw [|-|,thick] (-3,0.5) -- (0,0.5) node [midway,above] {\( L = 3[\text{m}] \)};
			\node [scale=1.4] at (-2,2.3) {\( \rho_1 \)};
			\node [scale=1.4] at (1.5,2.3) {\( \rho_2 \)};
		\end{scope}
	\end{tikzpicture}
\end{figure}